%!TEX TS-program = xelatex
\documentclass[12pt]{article}
\usepackage{geometry}
\geometry{verbose,letterpaper,tmargin=2.2cm,bmargin=2.2cm,lmargin=2.2cm,rmargin=2.2cm}
\usepackage[doublespacing]{setspace}
\usepackage[left]{lineno}
\renewcommand{\linenumberfont}{\normalfont\tiny}

% Figures at the end of the manuscript

% fonts
\usepackage{lmodern}

% authors and affiliations
\usepackage{authblk}
\renewcommand\Authfont{\scshape\normalsize}
\renewcommand\Affilfont{\itshape\small}

\usepackage{amssymb,amsmath}
\usepackage{ifxetex,ifluatex}
\ifnum 0\ifxetex 1\fi\ifluatex 1\fi=0 % if pdftex
  \usepackage[T1]{fontenc}
  \usepackage[utf8]{inputenc}
  \usepackage{textcomp} % provide euro and other symbols
\else % if luatex or xetex
  \usepackage{unicode-math}
  \defaultfontfeatures{Scale=MatchLowercase}
  \defaultfontfeatures[\rmfamily]{Ligatures=TeX,Scale=1}
\fi

% Use upquote if available, for straight quotes in verbatim environments
\IfFileExists{upquote.sty}{\usepackage{upquote}}{}
\IfFileExists{microtype.sty}{% use microtype if available
  \usepackage[]{microtype}
  \UseMicrotypeSet[protrusion]{basicmath} % disable protrusion for tt fonts
}{}
\makeatletter
\@ifundefined{KOMAClassName}{% if non-KOMA class
  \IfFileExists{parskip.sty}{%
    \usepackage{parskip}
  }{% else
    \setlength{\parindent}{0pt}
    \setlength{\parskip}{6pt plus 2pt minus 1pt}}
}{% if KOMA class
  \KOMAoptions{parskip=half}}
\makeatother
\usepackage{xcolor}
\IfFileExists{xurl.sty}{\usepackage{xurl}}{} % add URL line breaks if available
\IfFileExists{bookmark.sty}{\usepackage{bookmark}}{\usepackage{hyperref}}
\hypersetup{
  pdftitle={Paying colonization credit with forest management could accelerate the range shift of temperate trees under climate change},
  pdfauthor={Willian~Vieira and Isabelle~Boulangeat and Marie-Hélène~Brice and Robert
L.~Bradley and Dominique~Gravel},
  pdflang={en},
  pdfkeywords={Adaptive management, Assisted
migration, Resilience, Range dynamics, Transient
dynamics, Metapopulation, State and Transition Models},
colorlinks=true,
allcolors=[rgb]{0,0.4,0.5},
pdfcreator={LaTeX via pandoc}}
\urlstyle{same} % disable monospaced font for URLs






\usepackage{graphicx}
\makeatletter
\def\maxwidth{\ifdim\Gin@nat@width>\linewidth\linewidth\else\Gin@nat@width\fi}
\def\maxheight{\ifdim\Gin@nat@height>\textheight\textheight\else\Gin@nat@height\fi}
\makeatother
% Scale images if necessary, so that they will not overflow the page
% margins by default, and it is still possible to overwrite the defaults
% using explicit options in \includegraphics[width, height, ...]{}
\setkeys{Gin}{width=\maxwidth,height=\maxheight,keepaspectratio}
% Set default figure placement to htbp
\makeatletter
\def\fps@figure{htbp}
\makeatother


\setlength{\emergencystretch}{3em} % prevent overfull lines
\providecommand{\tightlist}{%
  \setlength{\itemsep}{0pt}\setlength{\parskip}{0pt}}

\setcounter{secnumdepth}{5}




\newlength{\cslhangindent}
\setlength{\cslhangindent}{1.5em}
\newenvironment{cslreferences}%
  {\setlength{\parindent}{0pt}%
  \everypar{\setlength{\hangindent}{\cslhangindent}}\ignorespaces}%
  {\par}

\title{Paying colonization credit with forest management could
accelerate the range shift of temperate trees under climate change}


\author[1,2]{Willian~Vieira\thanks{Corresponding author: willian.vieira@usherbrooke.ca}}
\author[3]{Isabelle~Boulangeat}
\author[4,5,2]{Marie-Hélène~Brice}
\author[1,6]{Robert L.~Bradley}
\author[1,2]{Dominique~Gravel}

\affil[1]{Département de biologie, Université de Sherbrooke, Sherbrooke,
Québec, Canada}
\affil[2]{Québec Centre for Biodiversity Sciences, McGill University,
Montréal, Québec, Canada}
\affil[3]{Université Grenoble Alpes, INRAE, LESSEM, St-Martin-d'Hères,
France}
\affil[4]{Jardin botanique de Montréal, Montréal, Québec, Canada}
\affil[5]{Institut de recherche en biologie végétale, Département de
Sciences Biologiques, Université de Montréal, Montréal, Québec, Canada}
\affil[6]{Centre d'étude de la forêt, Université du Québec à Montréal,
Montréal, Québec, Canada}


\date{}

\linenumbers

\begin{document}

% print all title page only if not double-blind
\maketitle
\small{{\bf Running title}: Management could accelerate tree range
shifts}\\\\
\small{{\bf Supporting information}: Additional supporting information can be found \href{https://willvieira.github.io/ms_STM-managed/suppInfo.pdf}{here}.}\\\\
\small{{\bf Acknowledgments}: We thank Daniel Houle for insightful
discussions on the effect of forest management practices, and his
comments on the early version of this manuscript. We are also grateful
to Steve Vissault for helping with the implementation of the model, and
to Antoine Becker Scarpitta for suggestions for improvement.}\\\\
\small{{\bf Data availability}: All the code and data used to reproduce
the analysis, figure and manuscript are stored as a research compendium
at \url{https://github.com/willvieira/ms_STM-managed}.}\\\\
\small{{\bf Funding}: This research was supported by the BIOS² NSERC
CREATE program, Strategic Project Grant from the Natural Sciences and
Engineering Council of Canada, and A joint Internship Grant from Mitacs
and Ouranos.}\\\\
\small{{\bf Authorship}: WV, IB and DG designed the study. WV and IB
developed the model. WV analyzed the simulations and wrote the first
draft of the manuscript. All authors contributed substantially to
revisions.}\\\\
\small{{\bf Conflicts of interest}: The authors declare no conflict of
interest.}\\\\
\small{{\bf Highlights}:\vspace{-1.5em}
\begin{itemize}
      \item We test the effect of sylvicultural practices in a forest
model under climate change.
      \item We assess how planting and harvesting can increase forest
resilience and range shift.
      \item Planting temperate trees is more efficient at increasing
response rate to climate change.
      \item Enrichment planting boreal stands is more efficient than
planting empty stands.
      \item Plantation increases resilience and coldward range shift to
keep up with climate change.
  \end{itemize}}
\begin{abstract}
The northward migration of several tree species ranges is likely to lag
behind climate change due to slow demography, competitive interactions,
and dispersal limitations. These will result in a colonization credit,
where suitable climate envelopes are left unoccupied, and extinction
debt, where tree stands persist at unsuitable climatic locations. While
the underlying mechanisms explaining the delayed range shift of forest
trees have been investigated, few studies have focused on how management
could overcome this lag. Here we extend a forest community state model
derived from the metapopulation theory and validated with over 40,000
forest inventory plots, to formulate how forest management can
accelerate the response of the boreal-temperate ecotone under warming
temperature. We first complete the model equations to represent how four
types of forest management may affect the transitions between four
forest states: Boreal, Temperate, Mixed and Regeneration. We then
simulated the potential of forest management to reduce colonization
credit and extinction debt using two complementary approaches to measure
the resilience and range shift of the boreal-temperate ecotone in
response to warming temperature. Our simulations reveal that paying the
colonization credit by planting temperate trees in a stand in
Regeneration or Boreal state are likely to i) reduce the return time to
equilibrium, ii) increase forest resilience, and iii) move the ecotone
towards colder temperatures. Surprisingly, harvesting boreal trees in
stands in Boreal or Mixed state were not effective to reduce extinction
debt and provide colonization opportunities for temperate trees. Our
results suggest that forest management related to planting actions could
help the boreal-temperate ecotone keep pace with climate change. Future
experiments are required to test these theoretical expectations and make
operational recommendations.
\end{abstract}
\hspace{1cm}\small{{\bf Keywords}: Adaptive management, Assisted
migration, Resilience, Range dynamics, Transient
dynamics, Metapopulation, State and Transition Models}



\hypertarget{introduction}{%
\section{Introduction}\label{introduction}}

There is a growing concern about how tree species will respond to
climate change, and how fast they can migrate to keep pace with climate
warming. Correlative statistical models have projected large range
shifts following temperature increases, such as the migration of plant
species hundreds of kilometers northward by the end of this century
(Malcolm et al. \protect\hyperlink{ref-Malcolm2002}{2002}, Mckenney et
al. \protect\hyperlink{ref-Mckenney2007}{2007}). While the range of
short-lived mobile species may keep pace with climate change (Chen et
al. \protect\hyperlink{ref-Chen2011}{2011}), the range of long-lived
tree species generally does not (Harsch et al.
\protect\hyperlink{ref-Harsch2009}{2009}, Zhu et al.
\protect\hyperlink{ref-Zhu2012}{2012}). In fact, trees of eastern North
America have shifted their range limits way bellow of the pace required
to keep up with temperature increases (Boisvert-Marsh et al.
\protect\hyperlink{ref-BoisvertMarsh2014}{2014},
\protect\hyperlink{ref-BoisvertMarsh2019}{2019}, Sittaro et al.
\protect\hyperlink{ref-Sittaro2017}{2017}). This mismatch between
climate conditions and forest community composition will likely lead to
maladaptation of trees to their environment, and therefore a possible
loss of future forest productivity (Aitken et al.
\protect\hyperlink{ref-Aitken2008}{2008}). Assessing the mechanisms
determining species range limits is, therefore, critical for formulating
adaptive management strategies (Becknell et al.
\protect\hyperlink{ref-Becknell2015a}{2015}).

Range limits of forest trees are driven by colonization and extinction
dynamics. The metapopulation theory predicts the boundary of a species'
range occurs where the colonization rate equals the extinction rate,
wherever habitat is available (Holt and Keitt
\protect\hyperlink{ref-Holt2000}{2000}). Derived from this theory,
Talluto et al. (\protect\hyperlink{ref-Talluto2017}{2017}) quantified
the colonization and extinction rates as a function of climate for 21
tree species in eastern North America and found that their distribution
is out of equilibrium with the current climate. Specifically, they found
a colonization credit at the leading edge of their range whereby
suitable habitat is left unoccupied, and an extinction debt at the
trailing edge whereby populations persist in unsuitable habitats. This
equilibrium mismatch is predicted to increase in the future, as the
range limits of temperate trees will barely shift northward due to their
slow demography and limited dispersal rates (Vissault et al.
\protect\hyperlink{ref-Vissault2020}{2020}).

Forest management provides an opportunity to reduce colonization credit
and extinction debt and, therefore, accelerate range shifts. Although
some management practices, such as assisted migration (Peters and
Darling \protect\hyperlink{ref-Peters1985a}{1985}), have been proposed
as a potential tool towards this end (\emph{e.g.} Gray et al.
\protect\hyperlink{ref-Gray2011}{2011}), there has been extensive debate
about its effectiveness with no definite conclusion (\emph{e.g.}
McLachlan et al. \protect\hyperlink{ref-McLachlan2007}{2007}, Ricciardi
and Simberloff \protect\hyperlink{ref-Ricciardi2009}{2009}, Schwartz et
al. \protect\hyperlink{ref-Schwartz2009}{2009}, Vitt et al.
\protect\hyperlink{ref-Vila2010}{2009}). The truth is, temperature is
warming and there is an increased need to adapt forest management
practices to consider future environmental conditions (Keenan
\protect\hyperlink{ref-Keenan2015}{2015}, Ameztegui et al.
\protect\hyperlink{ref-Ameztegui2018}{2018}). In the boreal forest in
Quebec, simulations indicate that if current management practices
persist, climate change will decrease the maximum sustainable harvest
yield due to the heightened frequency of fires, which prevents
individuals from reaching maturity (Forestier en Chef
\protect\hyperlink{ref-BureauduForestierenChef2020}{2020}). Changing the
current management strategies to reduce colonization credit and
extinction debt can be obtained through different silvicultural
approaches that trigger or modify some ecological processes. There are
basically two broad categories of actions: harvesting (removing
individuals) or planting trees (adding individuals). Large-scale
harvesting may reduce extinction debt by removing maladapted individuals
at the trailing edge, and also reduce colonization credit by reducing
competitive interactions at the leading edge (Leithead et al.
\protect\hyperlink{ref-Leithead2010}{2010}, Steenberg et al.
\protect\hyperlink{ref-Steenberg2013}{2013}, Brice et al.
\protect\hyperlink{ref-Brice2020}{2020}). Similarly, stand thinning
could improve the competitive ability and recruitment of certain tree
species that thrive in forest gaps. Alternatively, the planting of novel
species or genotypes in open areas, or enrichment planting in mature
stands (which increase the population of a tree species in a stand
before natural dynamics) could favor the desired successional pathways.
In the next section, we will develop in detail the link between forest
management and the ecological processes as we introduce the model.

In this paper, we will study how forest management can accelerate the
response of the boreal-temperate ecotone to climate warming. We first
extend a field-based model derived from metapopulation theory to
determine how four different management practices affect the
colonization and extinction processes driving tree range dynamics. Our
analysis is based on an empirical model which accounts for colonization
and extinction dynamics, along with competitive exclusion and invasion
processes, to predict how the boreal-temperate ecotone responds to
climate warming (Vissault et al.
\protect\hyperlink{ref-Vissault2020}{2020}). This model was initially
calibrated and validated with data from over 40,000 forest inventory
plots from eastern North America. We integrate the effects of
plantation, enrichment planting, harvest, and thinning on the
colonization and extinction dynamics of temperate deciduous and boreal
conifer stands.

We then assess the theoretical effectiveness of the four management
practices using two complementary approaches that quantify: (i) the
transient dynamics under equilibrium and (ii) the forest range shifts on
a lattice grid (Figure \ref{fig:concept}). Transient dynamics are
defined as the period a forest stand takes to reach a new equilibrium
after a temperature-increase (Hastings
\protect\hyperlink{ref-Hastings2004}{2004}). In dynamic models,
equilibrium is defined as the absence of change in a state variable over
time. We simulate an increase in temperature and analyze the effect of
forest management in five metrics charactherizing the transient dynamics
(Boulangeat et al. \protect\hyperlink{ref-Boulangeat2018}{2018}).
Initial resilience (\(-R_0\)) and asymptotic resilience (\(R_{\infty}\))
measure the rate of change near to the initial and final equilibriums
and are read as the system's reactivity and stability, respectively.
Exposure (\(\Delta_{state}\)) measures the degree to which the old and
new equilibrium's states differ, and sensitivity (\(\Delta_{time}\))
describes the amount of time needed to reach the new equilibrium.
Cumulative amount of changes (\(\int S(t)dt\)) combines all four metrics
described above to quantify the total amount of state and time in which
the system is out of equilibrium and therefore vulnerable. In the second
approach, we implement a stochastic and spatially explicit version of
the model to account for limited dispersal. We quantify how each of the
management practices accelerates the range shift of the boreal-temperate
ecotone in a landscape grid. Because of the lack of abundant data on
forest management across a climate gradient, we could not parametrize
and validade the extended model. Rather, these analyzes will serve as
references to guide future empirical studies by revealing the potential
effect of forest management in accelerating the response of forest to
climate warming and thus contribute to the advancement of adaptive
management practices.

\begin{figure}
\hypertarget{fig:concept}{%
\centering
\includegraphics{manuscript/img/concept.png}
\caption{Conceptual schema of the two approaches used to test the effect
of forest management on the response of forest to temperature increases.
(a) Redrawn from Boulangeat et al.
(\protect\hyperlink{ref-Boulangeat2018}{2018}). The spatially implicit
version of the model was used to investigate how forest management
affects the transient dynamics following temperature increases. Take,
for instance, a patch with environmental conditions that mainly favour
boreal species, the increase in temperature due to climate change will
now favour other species over the boreal ones. As a result, the boreal
state occupancy at equilibrium under the new climate (\(B_1\) at
\(t_1\)) will be lower than it was before climate change (\(B_0\) at
\(t_0\)). Five metrics can describe the transient phase between the old
and new equilibrium: initial resilience (\(-R_0\)), asymptotic
resilience (\(R_{\infty}\)), exposure (\(\Delta_{state}\)), sensitivity
(\(\Delta_{time}\)) and cumulative amount of changes (\(\int S(t)dt\)).
(b) The spatially explicit version of the model was used to study the
effect of forest management on the range shift of forest states
following climate change (CC) while accounting for limited dispersal of
trees and stochastic dynamics. The two lattice grids represent the
distribution of pure boreal, mixed, pure temperate, and regeneration
states along a gradient of temperature ranging from boreal dominant to
temperate dominant climate conditions. The cell size of the grids in
this figure was increased for visual clarity. The left and right
vertical black bars indicate the range limit between boreal and mixed,
and between mixed and temperate, respectively. The upper lattice shows
the distribution of forest states in equilibrium with climate before the
increase in temperature (initial state). The bottom lattice shows,
according to Vissault et al.
(\protect\hyperlink{ref-Vissault2020}{2020}), that after 150 years
following the increase in temperature, the mixed/temperate range limit
followed climate change (red arrow), but the boreal/mixed range limit
did not (faded red arrow). We use this scenario to study the potential
of forest management to accelerate the range shift of the
boreal-temperate ecotone towards colder
temperatures.}\label{fig:concept}
}
\end{figure}

\hypertarget{modelling-forest-range-limits-and-management-practices}{%
\section{Modelling forest range limits and management
practices}\label{modelling-forest-range-limits-and-management-practices}}

A classical model to study spatial dynamics at the regional spatial
scale comes from Levins' metapopulation theory (Levins
\protect\hyperlink{ref-Levins1969}{1969}). The theory is particularly
suitable to describe the mosaic of forest successional stages at the
landscape scale arising from natural disturbances and succession. The
model describes metapopulation as a set of patches that are either
occupied or empty and connected by dispersal. At this point, the model
is spatially implicit, meaning that dispersal is global and all patches
are connected equally. The dynamics of the metapopulation is given by
individuals arriving and establishing in empty patches through the
process of colonization (\(\alpha\)), and occupied patches becoming
empty through the process of extinction (\(\varepsilon\)):

\begin{equation}
\frac{dp}{dt} = \alpha p (1 - p) - \varepsilon p
\label{eq:metapop}\end{equation}

Where \(p\) is the proportion of occupied patches. We can further extend
this model to incorporate an environmental gradient by turning the
demographic parameters (\(\alpha\) and \(\varepsilon\)) into functions
of climate conditions. As a result, we can derive range limits as the
set of environmental conditions where the extinction rate equals the
colonization rate (Holt et al. \protect\hyperlink{ref-Holt2005}{2005}).
Relaxing the assumption of one single species dynamics, we can consider
multiple species competing for the same patches by having both
colonization and extinction parameters varying as a function of species
interactions (Gravel and Massol
\protect\hyperlink{ref-Gravel2020}{2020}). In this multi-species
setting, range limits are not only determined by climate, but also by
interactions that can either reduce or expand the northward limit
(Godsoe et al. \protect\hyperlink{ref-Godsoe2017}{2017}). The
theoretical model composed of differential equations can be made
spatially explicit, meaning every patch is located on a lattice and that
dispersal only occurs between neighboring patches. The spatially
explicit model allows us to account for the effect of dispersal
limitations when predicting the response of trees to climate warming.
Our model previously parameterized for eastern North American forests is
derived from this theory (Vissault et al.
\protect\hyperlink{ref-Vissault2020}{2020}).

Forest landscapes have been conceptualized as a dynamic mosaic of
different states for a long time (Picket and White
\protect\hyperlink{ref-PICKETT1985}{1985}). While the formal application
of Levins' metapopulation model over a climatic gradient is recent
(Talluto et al. \protect\hyperlink{ref-Talluto2017}{2017}), it builds on
key concepts formalized in previous forest dynamic models. Among the
first ones is the description of successional dynamics with a transition
probability matrix by Horn
(\protect\hyperlink{ref-horn1971adaptive}{1971}). Our approach described
bellow is somehow very similar, with the particularity that the
transition matrix is non-stationary over a climatic gradient and
conditional on state occupancy. Levin and Paine
(\protect\hyperlink{ref-Levin1974}{1974}) followed not long after with
with a model of disturbances and patch formation used to derive
steady-state distributions of different patch states. Forest gap models
like Jabowa were developped independently (Botkin et al.
\protect\hyperlink{ref-Botkin1972}{1972}) and later followed by
landscape models like Landis (Mladenoff et al.
\protect\hyperlink{ref-Mladenoff1996}{1996}) and its climate-dependent
variant Landis-II (Scheller and Mladenoff
\protect\hyperlink{ref-Scheller2004}{2004}). Such models, and other
descendants, differ significantly in implementation, scope and details,
but they all share the common feature that landscapes are composed of
patches subject to disturbances (extinction) and succession
(colonization, exclusion) between different states. Our motivation with
the Levins' approach was twofold : i) maintain mathematical tractability
to facilitate its analysis and ii) facilitate model calibration on
forest inventory data. Below we summarize the model conception and
calibration to ease the reading and refer to Vissault et al.
(\protect\hyperlink{ref-Vissault2020}{2020}) for a detailed description
and sensitivity analysis. We will then develop the integration of the
four management practices in the following section.

The State and Transition Model (STM) considers three discrete forest (or
occupied) states along a gradient of temperature: (B)oreal, (T)emperate,
and (M)ixedwood forest states; and the (R)egeneration (or empty) state
(Vissault et al. \protect\hyperlink{ref-Vissault2020}{2020}). The
colonization (\(\alpha\)) and extinction (\(\varepsilon\)) processes
drive the transitions between empty (R) and occupied (by either B, M, or
T) patches. The model describes species interaction through the
mechanisms of invasion and competitive exclusion. Invasion (\(\beta\))
happens when an occupied state type of pure boreal (B) or pure temperate
(T) is colonized by tree species from the opposite type, and becomes
then a mixed state (M). Competitive exclusion (\(\theta\)) drives the
transitions from a mixed forest state (M) to either state boreal (B) or
temperate (T), depending on the competitive ability of each of forest
types. The rate at which occurs each of these processes (\(\alpha\),
\(\varepsilon\), \(\beta\), and \(\theta\)) is specific to the forest
type and the local climatic conditions, and the resulting process is
dependent on the amount of the corresponding state in the landscape
(Figure \ref{fig:model} a).

The parameters describing transitions among states were calibrated using
over 40,000 plots from the eastern North American forest (Vissault et
al. \protect\hyperlink{ref-Vissault2020}{2020}). In this study, the
database incorporates data from the FIA in the United States (O'Connell
et al. \protect\hyperlink{ref-OConnell2007}{2007}), the Canadian
provinces of Québec, Ontario, and New Brunswick (Porter
\protect\hyperlink{ref-Porter2001}{2001}, Ontario Ministry of Natural
Resources \protect\hyperlink{ref-OMNR2014}{2014}, Ministere des
Ressources Naturelles \protect\hyperlink{ref-Naturelles2016}{2016}), as
well as a private forest company in Québec (Domtar). For each plot
(measured between 1960 and 2010) and each census, the forest states (B,
M, and T) were classified following their species composition. A stand
was classified as T whenever all boreal species were absent while at
least one of the following eight temperate species was present:
\emph{Prunus serotina}, \emph{Acer rubrum}, \emph{Acer saccharum},
\emph{Fraxinus americana}, \emph{Fraxinus nigra}, \emph{Fagus
grandifolia}, \emph{Ostrya virginiana}, and \emph{Tilia americana}.
Alternatively, a stand was classified as B whenever all temperate
species were absent while at least one of the following seven boreal
species was present: \emph{Picea mariana}, \emph{Picea glauca},
\emph{Picea rubens}, \emph{Larix laricina}, \emph{Pinus banksiana},
\emph{Abies balsamea}, \emph{Thuja occidentalis}. The stand was
classified as mixedwood (M) when both boreal and temperate species were
present. Therefore, T and B stands are inheritly pure compositions. The
stand was classified as regeneration (R) when the total basal area was
inferior to 5 m\(^2\) ha\(^{-1}\), irrespective of its species
composition. After classifying each plot year into one of the four
forest states, transitions were modelled as a function of local climate
conditions, namely mean annual temperature (MAT) and total annual
precipitation (TAP). Parameters of the non-linear multi-nomial models
were evaluated by maximum likelihood and a simulated annealing
optimization procedure. Note that this model avoids the presumption that
the point data is at equilibrium since it predicts the transition
between states rather than the distribution. Only permanent sampling
plots with a time interval within the 5-15 year range were used in the
parameterization (median time interval among plots \textasciitilde5
years). Furthermore, all disturbances such as fire, drought, and
outbreaks were included in the fitting of the STM; only managed plots
were excluded of the analysis to assure the four transition processes
were naturally induced. Part of the data not used in the calibration was
used to validate the predictions of the model. The parameters of the
model were validated by solving the model to equilibrium using current
climate conditions and comparing the model predictions to the current
forest distribution from the validation data. The accuracy of the STM in
predicting each of the four states given MAT and TAP ranged from 70\% to
98\% (Vissault et al. \protect\hyperlink{ref-Vissault2020}{2020}).

This simple State Transition Model allows one to predict the
distribution of forest community composition at the continental scale.
In the present study, we use the STM equations with their estimated
parameters to integrate the effects of four management practices. We are
aware of the theory that predicts species range limits as a process
derived from their local demographic vital rates (Ara\a'ujo and
Rozenfeld \protect\hyperlink{ref-Araujo2014a}{2014}, Normand et al.
\protect\hyperlink{ref-Normand2014}{2014}). Given that different species
within the same community have different demographic rates, their
response to climate change will likely generate different range shifts.
However, empirical studies have had little success in establishing the
link between the vital rate of tree species and their distribution
(Kunstler et al. \protect\hyperlink{ref-Kunstler2021}{2021}, Le Squin et
al. \protect\hyperlink{ref-LeSquin2021}{2021}). In addition, we can
expect that species within the same forest state will respond similarly
to each other compared to species in other states, regardless of the
demographic variance among the species of the same group. Since we are
interested in exploring how forest management affects forest range
limits, we chose to work beyond the species level to model general
management practices at the scale of forest community composition.
Therefore, we stress our study as a theoretical investigation to guide
future models and experimentation towards adaptive management practices.
In the next section, we detail the ecological assumptions and
mathematical formulation for each management practice implemented in the
model. Finally, with the extended model equations and estimated
parameters, we develop our two simulation approaches to test the effect
of forest management on transient dynamics and forest range shifts.

\begin{figure}
\hypertarget{fig:model}{%
\centering
\includegraphics[width=1\textwidth,height=\textheight]{manuscript/img/model_equation_fm.png}
\caption{Schema of the State and Transition Model adapted from Vissault
et al. (\protect\hyperlink{ref-Vissault2020}{2020}). Directional arrows
describe the colonization (\(\alpha\)), extinction (\(\varepsilon\)),
invasion (\(\beta\)), and competitive exclusion (\(\theta\)) processes
driving the transition between the four forest states: (R)egeneration,
(B)oreal, (T)emperate, and (M)ixedwood. The panel (b) summarises the
effect of increasing the intensity of forest management in each of the
four ecological processes. For instance, increasing plantation intensity
will increase the rate of transition from R to T and consequently
descrease the rate of change from R to B and from R to M. The values of
each of the 9 specific (process x state) parameters are shown in Figure
S7.}\label{fig:model}
}
\end{figure}

\hypertarget{adapted-forest-management-reducing-the-gap-between-potential-and-actual-forest-distribution}{%
\subsection{Adapted forest management: reducing the gap between
potential and actual forest
distribution}\label{adapted-forest-management-reducing-the-gap-between-potential-and-actual-forest-distribution}}

Given the predictions that the distribution of the boreal-temperate
ecotone may lag behind climate change (Vissault et al.
\protect\hyperlink{ref-Vissault2020}{2020}, Talluto et al.
\protect\hyperlink{ref-Talluto2017}{2017}), here we define and simulate
four management practices to test how they may reduce the gap between
potential and realized forest distribution with climate warming. The
four management practices implemented in the model are plantation and
enrichment planting to potentially reduce colonization credit, and
harvest and thinning to potentially reduce extinction debt. The
objective of these management practices is to favor the migration of
both the leading edge of temperate forest and the trailing edge of
boreal forest towards colder temperatures when the climate is suitable.

\hypertarget{forest-management-to-reduce-colonization-credit}{%
\subsubsection{Forest management to reduce colonization
credit}\label{forest-management-to-reduce-colonization-credit}}

Colonization of temperate species beyond the leading edge of their
distribution may depend on many factors such as climate conditions,
competitive ability, and seed sources through dispersion. The first
factor limiting the colonization of a population beyond its range is the
climate. Once the climate limitation is relaxed with climate warming,
species interactions such as competition for light may limit the
development of regenerating individuals (e.g.~Bianchi et al.
\protect\hyperlink{ref-Bianchi2018}{2018}). Finally, seed production is
a density-dependent process that, associated with the slow migration
rate of trees, contributes to the lack of colonization beyond the
population range limits. In the context of managing ecological
processes, some of these factors can be modified with forest management.
Here we model two management practices that may operate at different
spatial scales to simulate density-independent colonization: plantation
(i.e.~assisted migration) at the large spatial scale, and enrichment
planting at the local spatial scale. Plantation occurs in regeneration
states, while enrichment planting occurs in mature stands of the
alternative composition (e.g.~introducing temperate hardwoods in a
boreal stand). Following temperature increases, plantation and
enrichment planting of temperate species should overcome dispersal
limitation and the lack of seed sources and may increase the range shift
towards colder temperatures by colonizing stands beyond the current
distribution.

\hypertarget{plantation-of-temperate-stands}{%
\paragraph{Plantation of temperate
stands}\label{plantation-of-temperate-stands}}

In our model, the establishment of boreal, mixedwood or temperate forest
in regenerating stands depends on the colonization capacity of boreal
and temperate tree species (\(\alpha_B\) and \(\alpha_T\)) as well as
their abundance in the neighboring stands. The plantation practice is
modelled as an increase in the probability of regeneration stands to
become temperate forest stands \(P(T|R)\). A proportion \(p\) of
available stands in state R is thus converted into state T at each time
step. Only the remaining stands in state R (\(1-p\)) are allowed to
follow the natural colonization process. Plantation thus involves an
additional parameter \(p\) that modifies the following probabilities:

\begin{equation}
\begin{split}
&P(T|R) = [\alpha_T (T+M) \times (1-\alpha_B (B+M))] \times (1 - p) +  p \\[2pt]
&P(B|R) = [\alpha_B (B+M) \times (1-\alpha_T (T+M))] \times (1 - p) \\[2pt]
&P(M|R) = [\alpha_T (T+M) \times \alpha_B (B+M)] \times (1 - p)
\end{split}
\label{eq:plantation}\end{equation}

where \(p\) is the proportion of R stands that are planted per time
step. Note that when \(p=0\), the natural dynamics occurs and when
\(p=1\), \(P(T|R)=1\), \(P(B|R)=P(M|R)=0\).

\hypertarget{enrichment-planting-of-temperate-trees-on-boreal-stands}{%
\paragraph{Enrichment planting of temperate trees on boreal
stands}\label{enrichment-planting-of-temperate-trees-on-boreal-stands}}

Invasion of temperate species into boreal stands is a function of the
capacity of temperate forest trees to colonize boreal forest
\(\beta_T\), and the abundance of mixed and temperate in neighboring
stands. Invasion only applies to mature stands. Enrichment planting of
temperate species in boreal stands is modelled as an increase in the
probability of stands in state boreal to become mixedwood \(P(M|B)\).
Among stands in state B available to invasion, a proportion \(e\) is
directly converted to M. The colonization probability of temperate
species establishing in boreal stands after enrichment planting adds a
parameter \(e\) to the model:

\begin{equation}
P(M|B) = [(1- (\varepsilon \times (1 - h) + h)) \times \beta_T(T + M)] \times (1-e) + e
\label{eq:enrichplanting}\end{equation}

Where \(e\) is the proportion of mature stands in state B that are
enriched at each time step. Natural dynamics occurs when \(e=0\), while
direct conversion by forest management occurs when
\(P(M|B)= 1- (\varepsilon \times (1 - h) + h)\). Note that \(h\) is the
proportion of stands in state B that are harvested as explained in the
next section.

\hypertarget{forest-management-to-reduce-extinction-debt}{%
\subsubsection{Forest management to reduce extinction
debt}\label{forest-management-to-reduce-extinction-debt}}

Different ecological mechanisms can explain extinction debt caused by
the delayed response of forest trees to temperature increases. Slow
demographic rates along with dispersal limitations can delay the
response of species to environmental changes (Dullinger et al.
\protect\hyperlink{ref-Dullinger2012}{2012}). These life-history traits,
associated with source-sink dynamics (Schurr et al.
\protect\hyperlink{ref-Schurr2012}{2012}), can increase considerably the
extinction debt of tree populations following temperature increases. To
reduce this delayed response, unadapted species would have to disappear
and therefore make room for the new species that is better adapted to
the novel environmental conditions. Disturbance and competitive
exclusion are two ecological processes suitable to influence the rate of
extinction and, if well directed, reduce extinction debt. Here we chose
harvest and thinning, which is a partial harvest within a stand, as
complementary management practices that may accelerate disturbance and
competitive exclusion. Harvest of stands in state B has the same effect
than large spatial scale disturbances, such as fire, and transform a
proportion of B stands in a R state. Similarly, removal of boreal
species by selective thinning in stands of state M can increase the rate
at which temperate species can competitively exclude boreal species.
Both harvest and thinning are intended to open space and reduce the
proportion of boreal species, and therefore increase the likelihood of
temperate states to shift towards colder temperatures.

\hypertarget{harvest-of-boreal-stands}{%
\paragraph{Harvest of boreal stands}\label{harvest-of-boreal-stands}}

In the natural extinction model, stands in state B turn into a
regeneration state only after natural disturbances, occurring at a
probability \(\varepsilon\). Harvest is modelled as an increase in the
probability of boreal states to become regeneration states \(P(R|B)\). A
proportion \(h\) of mature stands in state B is converted into state R,
featuring the cut of all trees. This proportion of B stands is thus
excluded from following natural dynamics. Harvest thus involves an
additional parameter \(h\) that modifies the following probabilities:

\begin{equation}
\begin{split}
&P(R|B) = [\varepsilon \times (1 - h)] + h \\[2pt]
&P(M|B) = (1- (\varepsilon \times (1 - h) + h)) \times \beta_T(T + M)
\end{split}
\label{eq:harvestEq}\end{equation}

Where \(h\) is the proportion of stands in state B that are harvested at
each time step. If \(h=1\), no B stands will be maintained, and when
\(h=0\), only natural disturbance occurs.

\hypertarget{thinning-of-boreal-trees-in-mixedwood-stands}{%
\paragraph{Thinning of boreal trees in mixedwood
stands}\label{thinning-of-boreal-trees-in-mixedwood-stands}}

In the natural model, the transition from a mixed state M to either a
pure state (B or T) is driven by the instability of the state M
(\(\theta\)), and the competitive ratio between temperate and boreal
species (\(\theta_T\)). It means that the higher the instability
(\(\theta\)), the higher the probability of competitive exclusion, and
the winner is given the competitive ratio between temperate and boreal
species (\(\theta_T\)). Thinning of boreal species in M stands is
modelled as an increase of the probability of M stands to become state T
in two different ways (\(s_1\) and \(s_2\)). First, thinning of boreal
species can be translated into an increase in the instability of M
stands:

\begin{equation}
  \theta_{m} = [\theta \times (1 - s_1)] + s_1
\label{eq:thinningEq}\end{equation}

Second, selective thinning of boreal species can increase the
competitive ability of temperate species:

\begin{equation}
\theta_{T, m} = [\theta_{T} \times (1 - s_2)] + s_2
\label{eq:thinningEq2}\end{equation}

It is unclear if we need to distinguish between the two processes. The
rationale is that the proportion \(s_1\) of M stands that are managed
this way is directly converted into state T. It means that \(s_2\)
should at least be equal to \(s_1\). If thinning further boost the
competitivity (fitness) of temperate species, then \(s_2\) can be
greater than \(s_1\). For a parsimonious approach, it seems reasonable
to set \(s_1=s_2\). These modifications directly affect \(P(T|M)\) and
\(P(B|M)\):

\begin{equation}
\begin{split}
&\theta_{m} = [\theta \times (1 - s)] + s \\[2pt]
&\theta_{T, m} = [\theta_{T} \times (1 - s)] + s \\[2pt]
&P(T|M) = \theta_m \times \theta_{T,m} \times (1 - \varepsilon) \\[2pt]
&P(B|M) = \theta_m (1 - \theta_{T,m}) \times (1 - \varepsilon)
\end{split}
\label{eq:thinningEq3}\end{equation}

Where \(s\) is the proportion of undisturbed stands in state M where
thinning is applied per time step. When \(s=1\), \(P(T|M) = 1\) and
\(P(B|M) = 0\).

\hypertarget{simulation-analysis}{%
\subsection{Simulation analysis}\label{simulation-analysis}}

\hypertarget{analysis-of-the-transient-dynamics-under-climate-warming}{%
\subsubsection{Analysis of the transient dynamics under climate
warming}\label{analysis-of-the-transient-dynamics-under-climate-warming}}

We used the spatially implicit version of the STM at equilibrium with
current climate conditions to test the effect of forest management on
the transient dynamics following temperature increases. To do so, we
simulated an increase in temperature and focused on the dynamics of the
transient period of the four forest states until they reach the new
steady state. Steady state was considered as being reached when the
difference between two successive states prevalence was inferior to
\(10^{-7}\) for 10 consecutive steps. Each step in the model is equal to
5 years according to the initial parameterization of the model (Vissault
et al. \protect\hyperlink{ref-Vissault2020}{2020}). We characterized the
transient dynamics over a gradient of mean annual temperature ranging
from -2.61 to 5.07 \(^{\circ}\)C. Note that this approach quantifies the
model's local stability for a specific location defined by climatic
conditions. As a result, no spatially explicit dynamics like dispersal
are considered, and the transient metrics are calculated separately for
each location along the MAT gradient. This gradient corresponds to the
current temperature range along with the temperate-to-boreal forest
ecotone, and it is the reason we describe this gradient as ``initial
mean annual temperature''. This gradient can be visualized by drawing a
straight line from Montreal (\textasciitilde45.5 \(^{\circ}\) N) to
Chibougamau (\textasciitilde49.9 \(^{\circ}\) N), in Canada. While we
simulated temperature changes, TAP was kept constant to the mean value
extracted from the database (998.7 mm) because TAP has a relatively
small effect on model outputs compared to MAT (Vissault et al.
\protect\hyperlink{ref-Vissault2020}{2020}). Temperature increased by
0.09 \(^{\circ}\)C at each time step for the first 20 steps (100 years)
for a total increase of 1.8 \(^{\circ}\)C following the Representative
Concentration Pathway (RCP) scenario of 4.5, and then remained constant
until the model reached the steady state. As we used a linear increase
of temperature to represent the boreal-temperate ecotone (ranging from
-2.61 to 5.07 \(^{\circ}\)C) instead of a real landscape, the RCP
scenarios are based on the mean global projections (IPCC
\protect\hyperlink{ref-IPCC2013}{2013}). We further tested the RCP8.5
scenario and observed that the increase in the disturbance intensity
with warmer temperatures only shifted the reponse to higher values, but
did not change the overall interpretation compared to RC4.5 (results not
shown).

We characterized the transient phase after temperature increases using
five different metrics from Boulangeat et al.
(\protect\hyperlink{ref-Boulangeat2018}{2018}). The first two metrics
are the asymptotic and initial resilience as measures of local stability
derived from the Jacobian Matrix \(J\) at the new equilibrium (Arnoldi
et al. \protect\hyperlink{ref-Arnoldi2016}{2016}). \(J\) was numerically
calculated using the R package rootSolve (Soetaert
\protect\hyperlink{ref-Soetaert2009a}{2009}, Soetaert and Herman
\protect\hyperlink{ref-Soetaert2009}{2009}). The asymptotic resilience
(\(R_{\infty}\)) is the leading eigenvalue of \(J\), and quantifies the
asymptotic rate of return to equilibrium after small perturbation. The
more negative \(R_{\infty}\), the greater is the asymptotic rate of
change back to the equilibrium, and therefore the greater the resilience
of the system. Although the stability metrics are computed at the new
equilibrium, we can derive the initial reactivity of the system to
disturbance using algebra transformation of the matrix \(J\) (Neubert
and Caswell \protect\hyperlink{ref-Neubert1997}{1997}). Initial
resilience (\(-R_0\)), defined by Arnoldi et al.
(\protect\hyperlink{ref-Arnoldi2016}{2016}) as the inverse of initial
reactivity, is the leading eigenvalue of the following matrix:

\begin{equation}
M = \frac{-J + J^T}{2}
\label{eq:jacob}\end{equation}

Positive values of \(-R_0\) indicate a smooth transition to the new
equilibrium whereas negative values indicate reactivity, that is, an
initial amplification in the opposite direction to the final
equilibrium. The third metric is the exposure of the ecosystem states
(\(\Delta_{state}\)), defined by the euclidean distance between initial
and final state prevalence among the four states (Dawson et al.
\protect\hyperlink{ref-Dawson2011}{2011}). It reports the amount of
change the system will experience. The fourth metric is the return time
(\(\Delta_{time}\)) or ecosystem sensitivity, which is estimated by the
number of time steps of the transitory phase. A combination of the
previous metrics, it describes how long it takes to reach the new
equilibrium. The last metric is the cumulative amount of changes in the
transitory phase, or ecosystem vulnerability (Boulangeat et al.
\protect\hyperlink{ref-Boulangeat2018}{2018}). It is defined as the sum
of all changes in the states after climate warming and is obtained by
the integral of the states change over time (\(\int S(t)dt\)). It
combines all of the prior metrics to describe how much the system is
``out-of-equilibrium'' or vulnerable. These five metrics together can
summarize the multidimensionality of the response of a system to
external disturbances.

We used five distinct simulation scenarios: natural dynamics without
forest management, 0.25\% of plantation, 0.25\% of enrichment planting,
1\% of harvest, and 0.25\% of thinning, at an annual rate. The above
values were chosen to maintain a certain degree of realism. In the
Canadian province of Quebec, about 1\% of the forest territory is
harvested annually. Of this 1\% harvested, only 20 to 25\% is followed
by planting. To our knowledge, enrichment planting and thinning of a
specific species are more complex to operate and rarely used in Quebec
and should not overpass the other practices, hence we chose to analyze
the same amount as the plantation. To further quantify the effect of
increasing the intensity of forest management from 0 to 100\% for each
practice. For instance, increasing plantation to 100\% (\(p = 1\)) means
that all regeneration stands will become T. For that, we chose two
locations from the gradient of temperature in which forest management
had the most effect on the metrics of transient dynamics: -1 and 0
\(^{\circ}\)C MAT which represents the leading and trailing edge of the
ecotone.

\hypertarget{analysis-of-the-range-shift-under-climate-warming}{%
\subsubsection{Analysis of the range shift under climate
warming}\label{analysis-of-the-range-shift-under-climate-warming}}

Using the model equations with forest management, we created a spatially
explicit version of the model with an artificial landscape (lattice) to
account for explicit dispersal limitations and stochastic dynamics, to
test the capacity of forest management to accelerate the range shift of
the boreal-temperate ecotone towards colder temperatures. The landscape
is composed as a regular grid of 1698 by 170 cells where each cell
(approx. 300 x 300 meters) at each time step is occupied by one of the
four forest states (R, B, T or M). Given the average dispersal rate for
some temperate trees is in the range of 5-15 \(m \cdot yr^{-1}\)
(Ribbens et al. \protect\hyperlink{ref-Ribbens1994}{1994}), with maximum
dispersal rates estimated in the post-glacial period reaching 260
\(m \cdot yr^{-1}\) (Feurdean et al.
\protect\hyperlink{ref-Feurdean2013}{2013}), our 300 m grid has
sufficient distance to account for the rare long-distance dispersal
events. Sensitivity analysis showed that the range shift following
climate warming increased with larger grid cells (from 1 hectare to 2500
hectares), but the effect was stronger in cells larger than 100 hectares
(Figure S1). While the choice of the cell size affects the absolute
value of range shift, it does not affect the relative effect of the
different forest management strategies. Moreover, although the smaller
the cell the better we model dispersion, smaller cells are
computationally expensive. Therefore, the size of 300 x 300 m (9 ha) was
the best compromise between these two factors. The gradient of the
landscape grid was defined using the same MAT range as in the spatially
implicit model (-2.61 to 5.07 \(^{\circ}\)C) to represent the whole
ecotone from boreal to temperate dominant forest types, with a constant
TAP of 998.7 mm. The prevalence of each state at time \(t + 1\) was
calculated considering the stand composition of the eight neighboring
cells and the temperature and precipitation condition of the cell at
time \(t\). The state of the current cell at time \(t + 1\) was then
randomly drawn from the transition probabilities. The effect of climate
warming on the landscape dynamics was simulated by increasing
temperature of 0.09 \(^{\circ}\)C for each cell at each time step for
the first 20 steps (100 years; RCP4.5). We further performed simulations
using the RCP8.5 scenario, and the results are shown in Figure S6. The
spatially explicit version of the model was bind into an R package
stored on GitHub (Vieira \protect\hyperlink{ref-STManaged2020}{2020}).
We used the released version v2.0 of the package to run the simulations
for this article.

We ran three simulations to compare the relative importance of
temperature increases, forest management, and their interaction with the
equilibrium distribution in future climate conditions. The intensity of
the four management practices was the same as used in the first
approach, and they were equally applied across the landscape. The model
simulated 150 years of forest dynamics under three different scenarios:
(i) only climate change, (ii) only one forest management practice, and
(iii) climate change and one forest management practice at a time. These
``virtual experimental treatments'' allow to independently characterize
their independent effects and also their interaction. These three
simulation scenarios were then compared with current (\(T_0\)) and
future (\(T_1\)) forest distribution at equilibrium with climate as
reference points. For each simulation and reference points, we
quantified the boreal and the mixed/temperate occupancy over the
gradient of initial mean annual temperature (-2.61 to 5.07
\(^{\circ}\)C). This allowed us to visualize the response of state
occupancy to each simulation. In addition, we computed the average range
shift of state occupancy in mean annual temperature for each simulation,
taking the initial distribution at equilibrium with climate (\(T_0\)) as
the starting point. Range shift represents the shift of state occupancy
relative to the initial mean annual temperature. This approach allowed
us to quantify the displacement of the boreal-temperate ecotone in the
grid without the need of arbritary thresholds to define the range limits
of a forest type. Range shift was calculated as the difference in
initial mean annual temperature between the first and final step of a
simulation run for all values of state occupancy ranging from 0 and 1.
We removed extreme values of state occupancy (\(state_{occ} < 0.07\);
\(state_{occ} > 0.93\)) to avoid miscalculation of range shift as our
approach was imprecise in these extreme locations. This filter had
little effect on the median and quantiles of range shift (Figure S2).
Negative values of range shift indicate a displacement of the
distribution of a forest type towards colder temperatures, whereas
positive values indicate a displacement towards warmer temperatures.

Finally, as the chosen time scale (150 years) and management intensity
may not be large enough to detect the response of forests to temperature
increases and forest management, we ran the same configuration of
simulations while increasing both the time scale and the management
intensity. The running time of each simulation was increased to 250, 500
and 1000 years, and management intensity for all practices increased to
2, 5, 10 and 20\%. We replicated the simulations 15 times, while varying
the initial landscape for each simulation. Initial landscapes were
randomly generated, with the prevalence of each cell determined by the
MAT value across the gradient of the lattice grid.

\hypertarget{results}{%
\section{Results}\label{results}}

\hypertarget{effect-of-forest-management-on-transient-dynamics-under-climate-warming}{%
\subsection{Effect of forest management on transient dynamics under
climate
warming}\label{effect-of-forest-management-on-transient-dynamics-under-climate-warming}}

We characterized the transient dynamics following an increase of 1.8
\(^{\circ}\)C in temperature along the boreal-temperate ecotone.
Overall, all metrics peaked in two specific regions, indicating maximum
resilience at the transition between boreal and mixedwood
(\textasciitilde{} -1 \(^{\circ}\)C), and at the transition between
mixedwood and temperate dominant forest types (\textasciitilde{} 3
\(^{\circ}\)C; Figure \ref{fig:num-res1} a for reference). Plantation
and enrichment planting of temperate species, which simulate the payment
of colonization credit, were the only two practices affecting
significantly the transient dynamics following climate warming. The
effect of these two practices on the transient metrics was observed only
in the transitional region between boreal and mixedwood. Exposure
increased with enrichment planting in the boreal region (Figure
\ref{fig:num-res1} b), meaning that forest management promoted the shift
of forest states to a new equilibrium. The time for the forest to reach
the new equilibrium following climate warming (sensitivity) was reduced
by about 40 and 80\% with plantation and enrichment planting,
respectively (Figure \ref{fig:num-res1} c). The cumulative state changes
(Figure \ref{fig:num-res1} d) integrates the variation in both exposure,
sensitivity, and resilience into a single metric, ecologically
interpreted as ecosystem vulnerability. In the transition between boreal
and mixedwood states, where vulnerability is at its peak, plantation and
enrichment planting reduced vulnerability by 55 and 78\%, respectively.
In both transition regions between dominant forest types, asymptotic
resilience was close to zero, meaning a weak resilience of the system
due to its slow rate of change following a perturbation (Figure
\ref{fig:num-res1} e). In the same locations, initial resilience was at
its peak, meaning that the system is less reactive to a disturbance
(Figure \ref{fig:num-res1} f). This means that the forest ecosystem has
a slow reaction at the beginning and/or at the end of the transient
phase (see Figure \ref{fig:concept} a for a visual interpretation).
Enrichment planting was the only practice to change both resilience
metrics, doubling asymptotic resilience, and reducing initial resilience
by 13\%. Reducing colonization credit through plantation and enrichment
planting of temperate species were effective in changing the transient
dynamics under temperature increases, helping forest to keep pace with
climate change.

\begin{figure}
\hypertarget{fig:num-res1}{%
\centering
\includegraphics[width=0.9\textwidth,height=\textheight]{manuscript/img/num-result.png}
\caption{Expected occupancy of boreal and temperate-mixed states at
equilibrium with climate before (\(T_0\)) and after (\(T_1\))
temperature increases (RCP4.5) as a climatic reference (a). (b-f)
Transient dynamics following climate warming along the gradient of mean
annual temperature for five different scenarios: natural dynamics
without forest management, 0.25\% of plantation, 0.25\% of enrichment
planting, 1\% of harvest and 0.25\% of thinning. Transient dynamics are
described by (b) exposure or the shift of forest states to the new
equilibrium; (c) sensitivity or the time for the state reach equilibrium
after climate warming; (d) vulnerability or the cumulative amount of
state changes after temperature increases; (e) asymptotic resilience or
the rate in which the system recovery to equilibrium; and (f) initial
resilience or the reactivity of the system after temperature
increases.}\label{fig:num-res1}
}
\end{figure}

Given that the effect of forest management on the transient metrics was
stronger in the transitional region between boreal and mixedwood state
dominance (Figure \ref{fig:num-res1}), we selected two contrasting
locations in this region to evaluate the effect of increasing forest
management intensity on the transient metrics (Figure
\ref{fig:num-res2}). Enrichment planting and plantation remained the two
practices with the greatest effect on the transient metrics, increasing
exposure and resilience, and decreasing the return time (sensitivity) in
the boreal region (at -1\(^{\circ}\)C; figure \ref{fig:num-res2} a-c).
Moreover, the effect of these two practices was non-linear, thus a small
increase in management intensity had a large effect on the transient
metrics. For instance, a 20\% increase in enrichment planting will
increase exposure to 90\% to its maximum (Figure \ref{fig:num-res2} a),
and reduce asymptotic resilience (Figure \ref{fig:num-res2} b) and
sensitivity (Figure \ref{fig:num-res2} c) to 70\% of their maximum. The
increase in harvesting intensity of boreal stands also increased the
exposure and sensitivity of the system (Figure \ref{fig:num-res2} c).
Similarly, increasing thinning intensity in mixedwood stands increased
exposure and sensitivity (Figure \ref{fig:num-res2} d, f) but reduced
resilience (Figure \ref{fig:num-res2} e). Increasing management
intensity can accelerate forest response to climate change through
enrichment planting or plantation, but it can also delay this response
through harvesting and thinning. Initial resilience and cumulative state
changes are omitted in the Figure \ref{fig:num-res2}, and can be found
in the supporting information (Figure S3).

\begin{figure}
\hypertarget{fig:num-res2}{%
\centering
\includegraphics{manuscript/img/num-result_2.png}
\caption{Effect of increasing management intensity on the transient
metrics characterizes how the model responds to climate warming
(RCP4.5). The effect of increasing management intensity is observed on
two specific climate conditions represented by the initial mean annual
temperature: -1 (dominated by boreal; left panels) and 0 (boreal/mixed
state ecotone; right panels) regions. Transient dynamics are described
by (i) exposure or the shift of forest states to the new equilibrium;
(ii) asymptotic resilience or the rate at which the system recovers to
equilibrium; and (iii) sensitivity or the time for the state to reach
equilibrium after climate warming. Details on each metric are described
in Figure \ref{fig:num-res1}}\label{fig:num-res2}
}
\end{figure}

\hypertarget{effect-of-forest-management-on-range-limit-shift-under-climate-warming}{%
\subsection{Effect of forest management on range limit shift under
climate
warming}\label{effect-of-forest-management-on-range-limit-shift-under-climate-warming}}

We investigated how forest management affects the range limit shift
between the the boreal trailing edge and the mixed leading edge using
spatially explicit simulations accounting for dispersal limitations and
stochastic dynamics. Given the state distribution dominance at
equilibrium with current climate (light shaded area in Figure
\ref{fig:sim-result}), we expect climate warming to push the forest
distribution towards colder temperatures with a median range shift of
-1.8 \(^{\circ}\)C (which corresponds to the simulated temperature
increase, dark shaded area in Figure \ref{fig:sim-result} and dashed
line in Figure \ref{fig:sim-result2} b). After 150 years with no
management and no climate change, the boreal and temperate+mixed forest
dominance slightly shifted towards warmer temperatures with a median
range shift of 0.10 \(^{\circ}\)C, the same rate when plantation,
harvest, and thinning were applied (Figure \ref{fig:sim-result2} a).
Enrichment planting with no climate change shifted the dominance of the
boreal-temperate ecotone towards colder temperatures with a median range
shift of -0.03 \(^{\circ}\)C. After 150 years with climate warming
following the RCP4.5 scenario, the range of boreal and temperate+mixed
shifted only -0.53 \(^{\circ}\)C, contrary to the expected -1.8
\(^{\circ}\)C (Figure \ref{fig:sim-result2} b). Furthermore, we can
observe under RCP4.5 without forest management that the slope of the
transition between boreal and temperate+mixed forest dominance increased
with climate warming, meaning that the smooth transition observed at the
initial condition (light shaded area) became a more abrupt transition
between these two forest types (Figure \ref{fig:sim-result}). In this
RCP scenario, neither plantation, harvest, nor thinning had a
significant effect on range shift compared to the unmanaged scenario
(Figure \ref{fig:sim-result2} b). Enrichment planting was the single
practice to increase range shift towards colder temperature with a
median of -1.31 \(^{\circ}\)C. Reducing colonization credit, through
enrichment planting, increased the range shift of the boreal-temperate
ecotone when interacting with climate change, creating a smooth
transition between the dominance of these two forest types.

\begin{figure}
\hypertarget{fig:sim-result}{%
\centering
\includegraphics[width=0.79\textwidth,height=\textheight]{manuscript/img/sim-result_RCP4.5.png}
\caption{Boreal (left panels) and mixedwood/temperate (right panels)
occupancy across the landscape grid covering the boreal-temperate
ecotone. State occupancy is the proportion of that state for a given
location of initial mean annual temperature in the landscape grid. Note
that because we are more interested in the boreal/mixed range limit, we
chose to simplify the figure by considering the mixed and temperate
states together. Light and dark shaded areas are a reference of the
state occupancy in the landscape at equilibrium before and after
temperature increases, respectively. We ran our model for 150 years
(T150) under three alternative scenarios: only climate change (CC), only
forest management (FM), and climate change with forest management (CC +
FM) to assess their interactions. The results are the mean and 99\%
confidence intervals of 15 replicates. Management intensity was set to
0.25\% for plantation, thinning, and enrichment planting, and 1\% for
harvest. The climate change scenario was RCP 4.5.}\label{fig:sim-result}
}
\end{figure}

Simulation time and management intensity of figure \ref{fig:sim-result}
and \ref{fig:sim-result2} were kept small for the sake of realism, but
we further tested how increasing these two parameters will affect range
shift of the boreal-temperate ecotone. Overall, increasing the
simulation time increases range shift towards colder temperatures,
approaching the expected equilibrium under the RCP4.5 scenario (Figure
\ref{fig:sim-result3} a-c; Figure S4). After 250 years of simulation,
enrichment planting shifted the distribution of the boreal-temperate
ecotone with a median of -1.71 \(^{\circ}\)C, nearly reaching the
expected equilibrium of -1.8 \(^{\circ}\)C (Figure \ref{fig:sim-result3}
a). The remaining management practices did not have a strong effect on
range shift, with a shared median between plantation, harvest, and
thinning around -0.85 \(^{\circ}\)C, compared with -0.79 \(^{\circ}\)C
when no management was applied. After 500 years of simulation, both
enrichment planting and plantation differed from the other practices,
with a median range shift of -1.85 \(^{\circ}\)C and -1.43
\(^{\circ}\)C, respectively (Figure \ref{fig:sim-result3} b). After a
thousand years, enrichment planting remained stable for 500 years, and
all the other practices almost reached the expected equilibrium, with a
median range shift around -1.59 \(^{\circ}\)C (Figure
\ref{fig:sim-result3} b).

\begin{figure}
\hypertarget{fig:sim-result2}{%
\centering
\includegraphics[width=0.75\textwidth,height=\textheight]{manuscript/img/sim-result_3.png}
\caption{Summary of range shift relative to initial mean annual
temperature for (a) no climate change and (b) climate change under
RCP4.5 scenario. Range shift is the difference between the initial
(\(T_0\) at equilibrium) and final state distribution after 150 years of
simulation. Negative values of range shift indicate a change in forest
distribution towards colder temperature whereas positive values indicate
a change towards warmer temperature. The horizontal dashed line
represents the median expected range shift when model reaches the
equilibrium. Management intensity was set to 0.25\% for plantation,
thinning, and enrichment planting, and 1\% for
harvest.}\label{fig:sim-result2}
}
\end{figure}

Increasing management intensity of up to 20\% per year, while keeping
the simulations running for 150 years, had different effects according
to the four management practices (Figure \ref{fig:sim-result3}; Figure
S5). At an intensity of 5\%, enrichment planting nearly approached the
maximum range shift allowed by the landscape size, with a median range
shift of -3.22 \(^{\circ}\)C, increased to -3.26 and -3.30 \(^{\circ}\)C
for the 10 and 20\% intensity, respectively. Plantation also exceeded
the expected equilibrium at the intensity of 10 and 20\%, with a median
range shift of -2.05 and -3.05 \(^{\circ}\)C, respectively. Harvest was
the only practice to not increase both the boreal and the
temperate-mixed range shift at the same rate. While harvest increased
boreal range shift up to -3.33 \(^{\circ}\)C with 20\% management
intensity, temperate-mixed increased from -0.55 \(^{\circ}\)C (2\%) to
-0.64 \(^{\circ}\)C (20\%). Increasing thinning intensity did not
increase the range shift of the boreal-temperate ecotone towards colder
temperatures, with a stable range shift around -0.53 \(^{\circ}\)C.

\begin{figure}
\hypertarget{fig:sim-result3}{%
\centering
\includegraphics{manuscript/img/sim-result_4.png}
\caption{Summary of range shift relative to initial mean annual
temperature for different simulation times (a-c) and management
intensities (d-g). Range shift is the difference between the initial
(\(T_0\) at equilibrium) and final state distribution after (i) 150
years of simulation for the panels d-g and (ii) 250, 500, and 1000 years
for the panels a-c.~Negative values of range shift indicate a change
towards colder temperature whereas positive values indicate a change
towards warmer temperature. The horizontal dashed lines represent the
median expected range shift when model reaches the equilibrium for the
particular simulation set. Management intensity was set to 0.25\% for
plantation, thinning, and enrichment planting, and 1\% for harvest for
the panels a-c.~For the panels d-g, management intensity for all the
four practices was set to 2, 5, 10, or 20\%,
respectively.}\label{fig:sim-result3}
}
\end{figure}

\hypertarget{discussion}{%
\section{Discussion}\label{discussion}}

It is pressing to investigate how forest biomes will respond to climate
warming, and how forest management can mitigate the negative impacts of
this perturbation. We extended a simple and informative modelling
framework based on metapopulation theory that let us to (i) establish a
link between forest management and the ecological processes setting
range limits, and (ii) investigate the effect of forest management on
the response of the boreal-temperate ecotone to climate change. Our
study suggests, based on two complementary simulation techniques, that
forest management could help the boreal-temperate ecotone keep pace with
climate change. Paying colonization credit by enrichment planting of
temperate tree species in boreal forest stands, and the plantation of
temperate species in regenerating stands, are likely to increase forest
resilience, reduce the time to reach a new equilibrium, and increase
range limit shifts towards colder temperatures. This theoretical
investigation provides new opportunities to design future experiments
testing the potential of forest management to adapt to climate change.
It should guide forest managers to take into account both natural and
anthropogenic disturbances on forest dynamics.

\textbf{\emph{How can plantation and enrichment planting reduce
colonization credit?}}

Although climate change is expected to drive a shift in forest
composition by favoring temperate over boreal trees, the
boreal-temperate ecotone is lagging behind climate change
(Boisvert-Marsh et al. \protect\hyperlink{ref-BoisvertMarsh2014}{2014},
\protect\hyperlink{ref-BoisvertMarsh2019}{2019}, Vissault et al.
\protect\hyperlink{ref-Vissault2020}{2020}, Talluto et al.
\protect\hyperlink{ref-Talluto2017}{2017}). Similar results are found on
altitudinal gradients, where the slow dieback of \emph{Picea abies}
prevents the expansion of other species (Scherrer et al.
\protect\hyperlink{ref-Scherrer2020}{2020}). Our results suggest that
plantation and enrichment planting of temperate species on the boreal
region can increase the response of the boreal-temperate ecotone to
climate warming by reducing the transient period and increasing the
range shift towards colder temperatures. To date, few studies have
tested how assisted migration can shift trees' range limits. For
instance, modelling the plantation of tree species more suitable to
future climate is predicted to increase resilience indicators such as
carbon stocks and tree species diversity (Hof et al.
\protect\hyperlink{ref-Hof2017}{2017}), and therefore plantation is
assumed to increase tree range shift under climate change. Using the
same rationale, simulating the plantation of tree species in future
suitable enviroments was demonstrated to increase both biomass
productivity and species diversity in multiple scenarios of climate
change (Duveneck and Scheller
\protect\hyperlink{ref-Duveneck2015}{2015}). We found that enrichment
planting slightly increased asymptotic resilience, which indicates a
faster recovery to equilibrium after climate change (Figure
\ref{fig:num-res1}). This is similar to a modelling study that suggests
forest management had limited ability to increase resistance and
resilience under climate change (Duveneck and Scheller
\protect\hyperlink{ref-Duveneck2016}{2016}).

\textbf{\emph{Why is enrichment planting practice more efficient than
planting?}}

Enrichment planting of temperate trees into boreal areas had a stronger
effect on both reducing the transient period and increasing range shift
when compared with planting temperate in disturbed (empty) areas. This
is due to three different mechanisms. First, the intensity of forest
management in the model is relative to the abundance of a particular
forest type in the lanscape (Figure S8); hence 0.25\% of boreal stands
being enriched is much higher than 0.25\% of regeneration stands being
planted since the number of boreal stands is proportionally larger than
the number of regeneration stands. That explains the need to increase
planting intensity beyond 0.25\% to increase the boreal range shift
towards colder temperatures (Figure S5). Second, management practices
are not spatially organized. While enrichment planting is necessarily
applied on boreal stands (and thus in the colonization credit area),
planting is applied in regeneration stands that are evenly distributed
across the landscape, including the mixedwood and temperate regions.
Finally, while enrichment planting implies both an increase of temperate
trees and a reduction of boreal stands, plantation involves only an
increase of temperate stands. These results suggest that enrichment
planting in local gaps has the best potential compared to plantation to
assist forests keep pace with climate change. For northern temperate
forests with different levels of shade tolerance, tree recruitment was
more effective in the presence of local canopy gaps compared to
recruitment in open areas after clearcut (LePage et al.
\protect\hyperlink{ref-LePage2000}{2000}).

\textbf{\emph{Why does reducing colonization credit increase range shift
but reducing extinction debt does not?}}

Reducing extinction debt by increasing the frequency of disturbance
(natural or anthropogenic) is expected to drive range shift by
eliminating maladapted species that would persist for a long period, and
then create colonization opportunities for advancing species (Kuparinen
et al. \protect\hyperlink{ref-Kuparinen2010}{2010}, Renwick and Rocca
\protect\hyperlink{ref-Renwick2015}{2015}). Here intensifying
disturbance by increasing harvest of boreal stands did not affect the
rate of range shift after temperature increases. This result
corroborates with those of Vanderwel and Purves
(\protect\hyperlink{ref-Vanderwel2014}{2014}) who found that harvesting
boreal species amplifies transitions to early-successional forest type,
but has no effect on the range shift of boreal conifers. Similarly
expect for the disturbance intensity, Brice et al.
(\protect\hyperlink{ref-Brice2020}{2020}) also found that moderate
disturbances increased the probability of transition from mixedwood to
temperate stands but had a small effect on the transition from boreal to
mixedwood. Such a lack of effect on range shift may be explained by the
fact that most harvested boreal stands regenerate to boreal again due to
source-sink dynamics and the ecosystem internal memory such as seed
bank. In a field experiment, Reich et al.
(\protect\hyperlink{ref-Reich2015}{2015}) showed that the growth rate of
juvenile trees increased in their colder range and decreased in their
warmer range when exposed to above and belowground temperature
increases. In other words, temperate trees will perform better than
boreal trees in the transition between their ranges. Therefore, limited
dispersal rather than competition may be the primary factor contributing
for a lack of temperate colonization in harvested patches.

\textbf{\emph{Thinning increases temperate tree range expansion, but
does not affect boreal stands}}

We explored the hypothesis that selective harvesting of boreal tree
species (thinning) on stands in state M would increase the proportion of
stands in state T in the landscape, and therefore increase the regional
pool to favor the colonization of temperate trees into the boreal
region. Thinning indeed increased the proportion of temperate stands in
the mixedwood region by an increase in competitive exclusion
(\(\theta_{m}\) and \(\theta_{T, m}\)). Similar results have been shown
that harvest increased temperate species in the mixedwood region of
Quebec (Boulanger et al. \protect\hyperlink{ref-Boulanger2019}{2019},
Brice et al. \protect\hyperlink{ref-Brice2020}{2020}). However, our
model also show that thinning did not have any effect on the range limit
of boreal stands. In other words, temperate trees did not colonize
boreal stands, even with a increasing source pools. Such a lack of
temperate progression onto the boreal region may be explained by the
difficulty of temperate trees to settle in boreal stands due to priority
effects and unfavourable substrates (Solarik et al.
\protect\hyperlink{ref-Solarik2018}{2018},
\protect\hyperlink{ref-Solarik2020}{2020}). This effect is included in
the model indirectly through the invasion (mean \(\beta_{T}\) = 0.62)
and colonization (mean \(\alpha_{T}\) = 0.99) parameters associated with
the temperate stand. This may be the result of plant-soil feedbacks or
the importance of gaps for temperate tree regeneration. For instance,
regeneration of temperate species such as red maple and red oak has been
shown to be facilitated in forest gaps, while most boreal species showed
no difference (Leithead et al.
\protect\hyperlink{ref-Leithead2010}{2010}).

\textbf{\emph{Limitations and future perspectives}}

We have found that plantation and enrichment planting have the potential
to reduce colonization credit to help forests to keep pace with climate
change. However, further experiments are necessary as the four simulated
practices in our study are an approximation of real management
practices. For instance, we simulated thinning as selective logging
boreal species in favor of temperate species, while in practice,
thinning generally focuses on reducing stand density and maintaining
commercial species. Such density reduction is tricky to address with our
model because local abundances are not accounted for. There is generally
a mismatch between our simulations at the community stand resolution
with the management practices that occur from the individual to the
population level. Being aware of that caveat, we urge future modelling
studies to concomitantly represent forest dynamics at several
organizational levels, while including detailed management practices.
Individual-level models accounting for demographic rates are useful to
predict how local mechanisms such as species interaction can scale up to
determining species range limits (Ara\a'ujo and Rozenfeld
\protect\hyperlink{ref-Araujo2014a}{2014}, Normand et al.
\protect\hyperlink{ref-Normand2014}{2014}, Snell et al.
\protect\hyperlink{ref-Snell2014}{2014}). Moreover, forest-landscape
models and dynamic vegetation models can more accurately simulate the
migration process (Lehsten et al.
\protect\hyperlink{ref-Lehsten2019}{2019}). In our context,
individual-level models can test the effect of forest management on
growth, mortality, and regeneration, while a community-level model such
as ours helps better understand how the effect of management practices
scales up. We should also cautiously interpret the effect of climate
change as simulated here. Although it is predicted that drought
intensity will increase in the future and may drive how the forest will
respond to climate change (Greenwood et al.
\protect\hyperlink{ref-Greenwood2017}{2017}), we have simulated only
temperature warming, while precipitation remained constant. Some studies
have shown tree species to be more sensitive to an increase in drought
rather than temperature (\emph{e.g.} white spruce Andalo et al.
\protect\hyperlink{ref-Andalo2005}{2005}). Drought is, however, more a
pulse disturbance (or shock), having potential cumulative effects on
trees, and involving thresholds. Moreover, it should be investigated
with various frequencies and intensities. The present study rather shows
how forest management could help communities adapt to a continuous
change in the environment, mainly driven by changes in temperature.

We have provided evidence that management practices could help forest
communities cope with the rate at which climate change is occurring
across the southern half of Quebec. However, we can expect the final
outcome to be sensitive to the spatial distribution of different
practices. For instance, harvesting boreal stands nearby the leading
edge of the mixedwood distribution may create a synergy. On the other
hand, a 20\% harvest intensity had a strong effect on the range shift of
boreal forest, while the temperate range did not move (Figure
\ref{fig:sim-result3} g), showing that there are other factors more
important than the spatial distribution of the management practice. We
have simulated here the effect of four management practices alone in
order to distinguish the most effective and identify the potential
important mechanisms. However, the interaction between management
practices may have synergic or cancelling effects. Our simulations show
no effect of plantation and harvest on the range shift of the
boreal-temperate ecotone at a short time scale of 150 years (figure
\ref{fig:sim-result}). However, planting temperate trees after
harvesting boreal stands may overcome the limitations of these two
practices when applied individually, specially if these practices are
applied in particular locations such as in the transition zone. We
propose future studies should focus on integrating different spatial and
organizational forest models (e.g.~Talluto et al.
\protect\hyperlink{ref-talluto2016}{2016}), so that the link between a
management practice and the ecological processes can be better adjusted
and detailed according to its specific scale.

\hypertarget{references}{%
\section*{References}\label{references}}
\addcontentsline{toc}{section}{References}

\hypertarget{refs}{}
\begin{cslreferences}
\leavevmode\hypertarget{ref-Aitken2008}{}%
Aitken, S. N., S. Yeaman, J. A. Holliday, T. Wang, and S. Curtis-McLane.
2008. Adaptation, migration or extirpation: climate change outcomes for
tree populations. Evolutionary Applications 1:95--111.

\leavevmode\hypertarget{ref-Ameztegui2018}{}%
Ameztegui, A., K. A. Solarik, J. R. Parkins, D. Houle, C. Messier, and
D. Gravel. 2018. Perceptions of climate change across the Canadian
forest sector: The key factors of institutional and geographical
environment. PLoS ONE 13:1--18.

\leavevmode\hypertarget{ref-Andalo2005}{}%
Andalo, C., J. Beaulieu, and J. Bousquet. 2005. The impact of climate
change on growth of local white spruce populations in Québec, Canada.
Forest Ecology and Management 205:169--182.

\leavevmode\hypertarget{ref-Araujo2014a}{}%
Ara\a'ujo, M. B., and A. Rozenfeld. 2014. The geographic scaling of
biotic interactions. Ecography 37:406--415.

\leavevmode\hypertarget{ref-Arnoldi2016}{}%
Arnoldi, J. F., M. Loreau, and B. Haegeman. 2016. Resilience, reactivity
and variability: A mathematical comparison of ecological stability
measures. Journal of Theoretical Biology 389:47--59.

\leavevmode\hypertarget{ref-Becknell2015a}{}%
Becknell, J. M., A. R. Desai, M. C. Dietze, C. A. Schultz, G. Starr, P.
A. Duffy, J. F. Franklin, A. Pourmokhtarian, J. Hall, P. C. Stoy, M. W.
Binford, L. R. Boring, and C. L. Staudhammer. 2015. Assessing
Interactions Among Changing Climate, Management, and Disturbance in
Forests: A Macrosystems Approach. BioScience 65:263--274.

\leavevmode\hypertarget{ref-Bianchi2018}{}%
Bianchi, S., S. Hale, C. Cahalan, C. Arcangeli, and J. Gibbons. 2018.
Light-growth responses of Sitka spruce, Douglas fir and western hemlock
regeneration under continuous cover forestry. Forest Ecology and
Management 422:241--252.

\leavevmode\hypertarget{ref-BoisvertMarsh2014}{}%
Boisvert-Marsh, L., C. P. \a'e, and S. de Blois. 2014. Shifting with
climate? Evidence for recent changes in tree species distribution at
high latitudes. Ecosphere 5:1--33.

\leavevmode\hypertarget{ref-BoisvertMarsh2019}{}%
Boisvert-Marsh, L., C. P. \a'e, and S. de Blois. 2019. Divergent
responses to climate change and disturbance drive recruitment patterns
underlying latitudinal shifts of tree species. Journal of Ecology
107:1956--1969.

\leavevmode\hypertarget{ref-Botkin1972}{}%
Botkin, D. B., J. F. Janak, and J. R. Wallis. 1972. Some ecological
consequences of a computer model of forest growth. The Journal of
ecology:849--872.

\leavevmode\hypertarget{ref-Boulangeat2018}{}%
Boulangeat, I., J. C. Svenning, T. Daufresne, M. Leblond, and D. Gravel.
2018. The transient response of ecosystems to climate change is
amplified by trophic interactions. Oikos:1--12.

\leavevmode\hypertarget{ref-Boulanger2019}{}%
Boulanger, Y., D. Arseneault, Y. Boucher, S. Gauthier, D. Cyr, A. R.
Taylor, D. T. Price, and S. Dupuis. 2019. Climate change will affect the
ability of forest management to reduce gaps between current and
presettlement forest composition in southeastern Canada. Landscape
Ecology 34:159--174.

\leavevmode\hypertarget{ref-Brice2020}{}%
Brice, M.-H., S. Vissault, W. Vieira, D. Gravel, P. Legendre, and M.-J.
Fortin. 2020. Moderate disturbances accelerate forest transition
dynamics under climate change in the temperate--boreal ecotone of
eastern North America. Global Change Biology 26:4418--4435.

\leavevmode\hypertarget{ref-Chen2011}{}%
Chen, I.-C. C., J. K. Hill, R. Ohlemüller, D. B. Roy, and C. D. Thomas.
2011. Rapid range shifts of species associated with high levels of
climate warming. Science 333:1024--1026.

\leavevmode\hypertarget{ref-Dawson2011}{}%
Dawson, T. P., S. T. Jackson, J. I. House, I. C. Prentice, and G. M.
Mace. 2011. Supporting Online Material for Beyond Predictions :
Biodiversity Conservation in a Changing Climate. Science 86.

\leavevmode\hypertarget{ref-Dullinger2012}{}%
Dullinger, S., A. Gattringer, W. Thuiller, D. Moser, N. E. Zimmermann,
A. Guisan, W. Willner, C. Plutzar, M. Leitner, T. Mang, M. Caccianiga,
T. Dirnböck, S. Ertl, A. Fischer, J. Lenoir, J. C. Svenning, A. Psomas,
D. R. Schmatz, U. Silc, P. Vittoz, and K. Hülber. 2012. Extinction debt
of high-mountain plants under twenty-first-century climate change.
Nature Climate Change 2:619--622.

\leavevmode\hypertarget{ref-Duveneck2015}{}%
Duveneck, M. J., and R. M. Scheller. 2015. Climate-suitable planting as
a strategy for maintaining forest productivity and functional diversity.
Ecological Applications 25:1653--1668.

\leavevmode\hypertarget{ref-Duveneck2016}{}%
Duveneck, M. J., and R. M. Scheller. 2016. Measuring and managing
resistance and resilience under climate change in northern Great Lake
forests (USA). Landscape Ecology 31:669--686.

\leavevmode\hypertarget{ref-Feurdean2013}{}%
Feurdean, A., S. A. Bhagwat, K. J. Willis, H. J. B. Birks, H. Lischke,
and T. Hickler. 2013. Tree Migration-Rates: Narrowing the Gap between
Inferred Post-Glacial Rates and Projected Rates. PLoS ONE 8.

\leavevmode\hypertarget{ref-BureauduForestierenChef2020}{}%
Forestier en Chef. 2020. Intégration des changements climatiques et
développement de la capacité d'adaptation dans la détermination des
niveaux de récolte au Québec. Page 60. Gouvernement du Québec, Roberval,
Canada.

\leavevmode\hypertarget{ref-Godsoe2017}{}%
Godsoe, W., J. Jankowski, R. D. Holt, and D. Gravel. 2017. Integrating
Biogeography with Contemporary Niche Theory. Trends in Ecology and
Evolution 32:488--499.

\leavevmode\hypertarget{ref-Gravel2020}{}%
Gravel, D., and F. Massol. 2020. Toward a general theory of
metacommunity ecology. \emph{in} K. S. McCann and G. Gellner, editors.
Theoretical ecology: Concepts and applications. Oxford University Press.

\leavevmode\hypertarget{ref-Gray2011}{}%
Gray, L. K., T. Gylander, M. S. Mbogga, P. Y. Chen, and A. Hamann. 2011.
Assisted migration to address climate change: Recommendations for aspen
reforestation in western Canada. Ecological Applications 21:1591--1603.

\leavevmode\hypertarget{ref-Greenwood2017}{}%
Greenwood, S., P. Ruiz-Benito, J. Martínez-Vilalta, F. Lloret, T.
Kitzberger, C. D. Allen, R. Fensham, D. C. Laughlin, J. Kattge, G.
Bönisch, N. J. B. Kraft, and A. S. Jump. 2017. Tree mortality across
biomes is promoted by drought intensity, lower wood density and higher
specific leaf area. Ecology Letters 20:539--553.

\leavevmode\hypertarget{ref-Harsch2009}{}%
Harsch, M. A., P. E. Hulme, M. S. McGlone, and R. P. Duncan. 2009. Are
treelines advancing? A global meta-analysis of treeline response to
climate warming. Ecology Letters 12:1040--1049.

\leavevmode\hypertarget{ref-Hastings2004}{}%
Hastings, A. 2004. Transients: The key to long-term ecological
understanding? Trends in Ecology and Evolution 19:39--45.

\leavevmode\hypertarget{ref-Hof2017}{}%
Hof, A. R., C. C. Dymond, and D. J. Mladenoff. 2017. Climate change
mitigation through adaptation: the effectiveness of forest
diversification by novel tree planting regimes. Ecosphere 8:e01981.

\leavevmode\hypertarget{ref-Holt2000}{}%
Holt, R. D., and T. H. Keitt. 2000. Alternative causes for range limits:
a metapopulation perspective. Ecology Letters 3:41--47.

\leavevmode\hypertarget{ref-Holt2005}{}%
Holt, R. D., T. H. Keitt, M. a Lewis, B. a Maurer, and M. L. Taper.
2005. Theoretical models of species' borders: single species approaches.
Oikos 108:18--27.

\leavevmode\hypertarget{ref-horn1971adaptive}{}%
Horn, H. S. 1971. The adaptive geometry of trees. Princeton university
press.

\leavevmode\hypertarget{ref-IPCC2013}{}%
IPCC. 2013. Climate change 2013: The physical science basis.
Contribution of working group I to the fifth assessment report of the
intergovernmental panel on climate change 1535.

\leavevmode\hypertarget{ref-Keenan2015}{}%
Keenan, R. J. 2015. Climate change impacts and adaptation in forest
management: a review. Annals of Forest Science 72:145--167.

\leavevmode\hypertarget{ref-Kunstler2021}{}%
Kunstler, G., A. Guyennon, S. Ratcliffe, N. Rüger, P. Ruiz-Benito, D. Z.
Childs, J. Dahlgren, A. Lehtonen, W. Thuiller, C. Wirth, M. A. Zavala,
and R. Salguero-Gomez. 2021. Demographic performance of European tree
species at their hot and cold climatic edges. Journal of Ecology
109:1041--1054.

\leavevmode\hypertarget{ref-Kuparinen2010}{}%
Kuparinen, A., O. Savolainen, and F. M. Schurr. 2010. Increased
mortality can promote evolutionary adaptation of forest trees to climate
change. Forest Ecology and Management 259:1003--1008.

\leavevmode\hypertarget{ref-Lehsten2019}{}%
Lehsten, V., M. Mischurow, E. Lindström, D. Lehsten, and H. Lischke.
2019. LPJ-GM 1.0: simulating migration efficiently in a dynamic
vegetation model. Geoscientific Model Development 12:893--908.

\leavevmode\hypertarget{ref-Leithead2010}{}%
Leithead, M. D., M. Anand, and L. C. R. Silva. 2010. Northward migrating
trees establish in treefall gaps at the northern limit of the
temperate-boreal ecotone, Ontario, Canada. Oecologia 164:1095--1106.

\leavevmode\hypertarget{ref-LePage2000}{}%
LePage, P. T. T., C. D. Canham, K. D. Coates, and P. Bartemucci. 2000.
Seed abundance versus substrate limitation of seedling recruitment in
northern temperate forests of British Columbia. Canadian Journal of
Forest Research 30:415--427.

\leavevmode\hypertarget{ref-LeSquin2021}{}%
Le Squin, A., I. Boulangeat, and D. Gravel. 2021. Climate-induced
variation in the demography of 14 tree species is not sufficient to
explain their distribution in eastern North America. Global Ecology and
Biogeography 30:352--369.

\leavevmode\hypertarget{ref-Levin1974}{}%
Levin, S. A., and R. T. Paine. 1974. Disturbance, patch formation, and
community structure. Proceedings of the National Academy of Sciences
71:2744--2747.

\leavevmode\hypertarget{ref-Levins1969}{}%
Levins, R. 1969. Some Demographic and Genetic Consequences of
Environmental Heterogeneity for Biological Control. Bulletin of the
Entomological Society of America 15:237--240.

\leavevmode\hypertarget{ref-Malcolm2002}{}%
Malcolm, J. R., A. Markham, R. P. Neilson, and M. Garaci. 2002.
Estimated migration rates under scenarios of global climate change.
Journal of Biogeography 29:835--849.

\leavevmode\hypertarget{ref-Mckenney2007}{}%
Mckenney, D. W., J. H. Pedlar, K. Lawrence, M. F. Hutchinson, D. W. M.
C. Kenney, and K. Campbell. 2007. Potential Impacts of Climate Change on
the Distribution of North American Trees. BioScience 57:939--948.

\leavevmode\hypertarget{ref-McLachlan2007}{}%
McLachlan, J. S., J. J. Hellmann, and M. W. Schwartz. 2007. A framework
for debate of assisted migration in an era of climate change.
Conservation Biology 21:297--302.

\leavevmode\hypertarget{ref-Naturelles2016}{}%
Ministere des Ressources Naturelles. 2016. Norme d'inventaire
ecoforestier: placettes-echantillons temporaires. Direction des
inventaires forestier, Ministère des Ressources naturelles,Québec.

\leavevmode\hypertarget{ref-Mladenoff1996}{}%
Mladenoff, D., G. Host, J. Boeder, and T. Crow. 1996. A spatial model of
forest landscape disturbance, succession and management. GIS and
Environmental Modeling: progress and research issues. Ft. Collins,
CO:175--179.

\leavevmode\hypertarget{ref-Neubert1997}{}%
Neubert, M. G., and H. Caswell. 1997. Alternatives to resilience for
measuring the responses of ecological systems to perturbations. Ecology
78:653--665.

\leavevmode\hypertarget{ref-Normand2014}{}%
Normand, S., N. E. Zimmermann, F. M. Schurr, and H. Lischke. 2014.
Demography as the basis for understanding and predicting range dynamics.
Ecography 37:1149--1154.

\leavevmode\hypertarget{ref-OConnell2007}{}%
O'Connell, M. B., E. B. LaPoint, J. A. Turner, T. Ridley, D. Boyer, A.
Wilson, K. L. Waddell, and B. L. Conkling. 2007. The forest inventory
and analysis database: Database description and users forest inventory
and analysis program. US Department of Agriculture, Forest Service.

\leavevmode\hypertarget{ref-OMNR2014}{}%
Ontario Ministry of Natural Resources. 2014. Permanent Sample Plot and
Permanent Growth Plot Reference Manual. Ontario Ministry of Natural
Resources.

\leavevmode\hypertarget{ref-Peters1985a}{}%
Peters, R. L., and J. D. S. Darling. 1985. The greenhouse effect and
nature reserves. BioScience 35:707--717.

\leavevmode\hypertarget{ref-PICKETT1985}{}%
Picket, S. T. A., and P. S. White. 1985. The Ecology of Natural
Disturbance and Patch Dynamics. Pages 385--455 (S. T. A. PICKETT, P. S.
B. T. -The Ecology of Natural Disturbance WHITE, and P. Dynamics, Eds.).
Academic Press, San Diego.

\leavevmode\hypertarget{ref-Porter2001}{}%
Porter, K. B. 2001. Base de données sur les placettes d'échantillonnage
permanentes du Nouveau-Brunswick (PSPDB v1. 0): guide de l'utilisateur
et analyse. Fredericton, N.-B.: Centre de foresterie de l'Atlantique.

\leavevmode\hypertarget{ref-Reich2015}{}%
Reich, P. B., K. M. Sendall, K. Rice, R. L. Rich, A. Stefanski, S. E.
Hobbie, and R. A. Montgomery. 2015. Geographic range predicts
photosynthetic and growth response to warming in co-occurring tree
species. Nature Climate Change 5:148--152.

\leavevmode\hypertarget{ref-Renwick2015}{}%
Renwick, K. M., and M. E. Rocca. 2015. Temporal context affects the
observed rate of climate-driven range shifts in tree species. Global
Ecology and Biogeography 24:44--51.

\leavevmode\hypertarget{ref-Ribbens1994}{}%
Ribbens, E., J. A. Silander Jnr, and S. W. Pacala. 1994. Seedling
recruitment in forests: Calibrating models to predict patterns of tree
seedling dispersion. Ecology 75:1794--1806.

\leavevmode\hypertarget{ref-Ricciardi2009}{}%
Ricciardi, A., and D. Simberloff. 2009. Assisted colonization is not a
viable conservation strategy. Trends in Ecology and Evolution
24:248--253.

\leavevmode\hypertarget{ref-Scheller2004}{}%
Scheller, R. M., and D. J. Mladenoff. 2004. A forest growth and biomass
module for a landscape simulation model, LANDIS: design, validation, and
application. Ecological modelling 180:211--229.

\leavevmode\hypertarget{ref-Scherrer2020}{}%
Scherrer, D., Y. Vitasse, A. Guisan, T. Wohlgemuth, and H. Lischke.
2020. Competition and demography rather than dispersal limitation slow
down upward shifts of trees' upper elevation limits in the Alps. Journal
of Ecology:1--15.

\leavevmode\hypertarget{ref-Schurr2012}{}%
Schurr, F. M., J. Pagel, J. S. Cabral, J. Groeneveld, O. Bykova, R. B.
O'Hara, F. Hartig, W. D. Kissling, H. P. Linder, G. F. Midgley, B.
Schröder, A. Singer, N. E. Zimmermann, R. B. O. Hara, F. Hartig, W. D.
Kissling, H. P. Linder, G. F. Midgley, J. W. G.-u. Frankfurt, and F.
Main. 2012. How to understand species' niches and range dynamics: A
demographic research agenda for biogeography. Journal of Biogeography
39:2146--2162.

\leavevmode\hypertarget{ref-Schwartz2009}{}%
Schwartz, M. W., J. J. Hellmann, and J. S. McLachlan. 2009. The
precautionary principle in managed relocation is misguided advice.
Trends in Ecology \& Evolution 24:474.

\leavevmode\hypertarget{ref-Sittaro2017}{}%
Sittaro, F., A. Paquette, C. Messier, and C. A. Nock. 2017. Tree range
expansion in eastern North America fails to keep pace with climate
warming at northern range limits. Global Change Biology:1--10.

\leavevmode\hypertarget{ref-Snell2014}{}%
Snell, R. S., A. Huth, J. E. M. S. Nabel, G. Bocedi, J. M. J. Travis, D.
Gravel, H. Bugmann, a. G. Guti\a'errez, T. Hickler, S. I. Higgins, B.
Reineking, M. Scherstjanoi, N. Zurbriggen, and H. Lischke. 2014. Using
dynamic vegetation models to simulate plant range shifts. Ecography
37:1184--1197.

\leavevmode\hypertarget{ref-Soetaert2009a}{}%
Soetaert, K. 2009. rootSolve: Nonlinear root finding, equilibrium and
steady-state analysis of ordinary differential equations (R package
v1.6).

\leavevmode\hypertarget{ref-Soetaert2009}{}%
Soetaert, K., and P. M. J. Herman. 2009. A Practical Guide to Ecological
Modelling. Using R as a Simulation Platform. Page 372. Springer.

\leavevmode\hypertarget{ref-Solarik2020}{}%
Solarik, K. A., K. Cazelles, C. Messier, Y. Bergeron, and D. Gravel.
2020. Priority effects will impede range shifts of temperate tree
species into the boreal forest. Journal of Ecology 108:1155--1173.

\leavevmode\hypertarget{ref-Solarik2018}{}%
Solarik, K. A., C. Messier, R. Ouimet, Y. Bergeron, and D. Gravel. 2018.
Local adaptation of trees at the range margins impacts range shifts in
the face of climate change. Global Ecology and Biogeography
27:1507--1519.

\leavevmode\hypertarget{ref-Steenberg2013}{}%
Steenberg, J. W. N., P. N. Duinker, and P. G. Bush. 2013. Modelling the
effects of climate change and timber harvest on the forests of central
Nova Scotia, Canada. Annals of Forest Science 70:61--73.

\leavevmode\hypertarget{ref-talluto2016}{}%
Talluto, M. V., I. Boulangeat, A. Ameztegui, I. Aubin, D. Berteaux, A.
Butler, F. Doyon, C. R. Drever, M.-J. J. Fortin, T. Franceschini, J.
Li\a'enard, D. McKenney, K. A. Solarik, N. Strigul, W. Thuiller, and D.
Gravel. 2016. Cross-scale integration of knowledge for predicting
species ranges: a metamodelling framework. Global Ecology and
Biogeography 25:238--249.

\leavevmode\hypertarget{ref-Talluto2017}{}%
Talluto, M. V., I. Boulangeat, S. Vissault, W. Thuiller, and D. Gravel.
2017. Extinction debt and colonization credit delay range shifts of
eastern North American trees. Nature Ecology \& Evolution 1:0182.

\leavevmode\hypertarget{ref-Vanderwel2014}{}%
Vanderwel, M. C., and D. W. Purves. 2014. How do disturbances and
environmental heterogeneity affect the pace of forest distribution
shifts under climate change? Ecography 37:10--20.

\leavevmode\hypertarget{ref-STManaged2020}{}%
Vieira, W. 2020, May. willvieira/STManaged: Stable version used in the
manuscript.

\leavevmode\hypertarget{ref-Vissault2020}{}%
Vissault, S., M. V. Talluto, I. Boulangeat, and D. Gravel. 2020. Slow
demography and limited dispersal constrain the expansion of
north-eastern temperate forests under climate change. Journal of
Biogeography 47:2645--2656.

\leavevmode\hypertarget{ref-Vila2010}{}%
Vitt, P., K. Havens, and O. Hoegh-Guldberg. 2009. Assisted migration:
part of an integrated conservation strategy. Trends in Ecology \&
Evolution 24:473--474.

\leavevmode\hypertarget{ref-Zhu2012}{}%
Zhu, K., C. W. Woodall, and J. S. Clark. 2012. Failure to migrate: Lack
of tree range expansion in response to climate change. Global Change
Biology 18:1042--1052.
\end{cslreferences}


\newpage


\end{document}